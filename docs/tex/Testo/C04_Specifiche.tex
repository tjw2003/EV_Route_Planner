\section{Introduzione specifiche} \hypertarget{section::\theHsection}
In questo capitolo analizzeremo le specifiche funzionali e non funzionali della applicazione da sviluppare.

La priorità di una funzione è scelta in base a quante funzioni dipendono da essa:

\begin{enumerate}
\item \textbf{Alta}: è fondamentale implementarla prima di procedere con il macroblocco delle funzioni con
media priorità. Sono funzioni da implementare principalmente l’una in serie con l’altra.
\item \textbf{Media}: è fondamentale implementarla prima di procedere con il macroblocco delle funzioni
con bassa priorità. È possibile implementare parallelamente queste funzioni.
\item \textbf{Bassa priorità}: da loro non dipende nessuna funzione.
\end{enumerate}

\section{Specifiche funzionali} \hypertarget{section::\theHsection}
Le specifiche funzionali dell'applicazione sono le seguenti:
\begin{enumerate}
\item \textbf{Calcolo del percorso}: una volta fornito la mappa delle colonnine e il percorso da un punto A ad un punto B l'app deve rielaborare i dati in modo da calcolare il nuovo percorso comprensivo di eventuali soste.
\item \textbf{Iscrizione utente}: l'applicazione deve poter permettere all'utente l'iscrizione in modo da avere ulteriori servizi.
\item \textbf{Modifica dati utente}: l'applicazione deve poter permettere all'utente la modifica dei propri dati all'interno del proprio profilo.
\item \textbf{Esportazione tragitto}: quando l'applicazione restituisce un percorso all'utente, questi deve essere in grado di esportarlo in un formato opportuno.
\item \textbf{Visualizzazioni statistiche}: l'utente dal proprio profilo è in grado di visualizzare le statistiche relative ai propri viaggi.
\item \textbf{Mostra alternative percorso}: l'applicazione può mostrare all'utente le alternative al percorso precedentemente calcolato.
\item \textbf{Memorizzazione luoghi preferiti}: l'utente è in grado di salvare i percorsi preferiti nel proprio profilo
\item \textbf{Visualizzazione percorso}: l'utente è in grado di vedere il percorso calcolato dall'algoritmo su una mappa.
\item \textbf{Visualizza dati utente}: l'utente è in grado di vedere i propri dati nell'apposito profilo utente.
\end{enumerate}

Sono riportati in Tabella 4.1 tutte le specifiche funzionali con le relative priorità.

\begin{table}[h]
\centering
\begin{tabular}{|c|l|c|c|}
\hline
Codice & Nome                            & Priorità & Implementato \\ \hline
R1     & Calcolo percorso                & A        & SI           \\ \hline
R2     & Iscrizione utente               & M        & NO           \\ \hline
R3     & Modifica dati utente            & B        & NO           \\ \hline
R4     & Esportazione tragitto           & M        & NO           \\ \hline
R5     & Visualizzazione statistiche     & B        & NO           \\ \hline
R6     & Mostra alternative percorso     & A        & NO           \\ \hline
R7     & Memorizzazione luoghi preferiti & M        & NO           \\ \hline
R8     & Visualizza percorso             & A        & SI           \\ \hline
R9     & Visualizza dati utente          & M        & NO           \\ \hline
\end{tabular}
\caption{Specifiche funzionali, priorità e implementazione}
\end{table}